

First, I would like to thank my advisor, Prof. Isobel Ojalvo, for her mentorship at each stage of this journey. Isobel has provided unparalleled insight and guidance at many junctions, while also giving me the freedom and space to learn and explore the many facets of high-energy physics research. Her dedication to her research group and generosity in making time for me when I needed it, while managing a schedule that seems to require more than 24 hours a day, has been incredible.

I would also like to thank Prof. Jim Olsen and Prof. Mariangela Lisanti for being my thesis committee members, as well as Prof. Dan Marlow for being the second reader of this thesis. I'd also like to thank Prof. Chris Tully for his guidance over the years and support in living abroad at CERN.

A huge thanks goes to Pallabi Das, who has been my mentor and collaborator on many projects. I remain inspired by her scientific acuity and her fortitude and optimism in the face of crushing pressure. As I have said, ``I would go to battle for Pallabi." I would also like to thank Alexander Savin for his tireless and dedicated mentorship on the Phase-2 project, and patience in explaining the intricacies of the Calo Layer-1 Trigger. I am also thankful to have worked with such a supportive and dedicated analysis team: Pallabi Das, Pieter Everaerts, Ho-Fung Tsoi, Anagha Aravind, Steffi Bower, and Hichem Bouchamaoui. 

I would also like to thank the more senior members of the Princeton CMS group for their mentorship and support. Andrew Loeliger provided invaluable support on all things software and computing. Adrian Alan Pol advised a project on Vivado HLS for the TAC-HEP program. Sam Higginbotham and Kelvin Mei gave me a warm welcome to Princeton and the high-energy physics group. Among my peers, I am honored to include Gillian Kopp, Bennett Greenberg, Ashling Quinn, and Elliott Kauffman. These people make working with the Princeton CMS group a joy.

I'd also like to thank the administrative members and staff of the Princeton Physics department, in particular Kim Dawidowski, Katherine Lamos, Kate Brosowsky,  Lisa Scalice, Regina Savadge, and Jennifer Bornkamp. They play integral roles in the department and research groups, and are wonderful people to see around Jadwin.

I tried to avoid getting sappy in the main text, but here we go, as I would like to thank my friends. Gillian, where do I start: we went through undergrad, grad school, COVID, and relocating to Switzerland together. Thanks for the support and companionship through thick and thin. Liz, it has been great to learn and grow throughout grad school with you, and your commitment to staying true to yourself is inspirational and grounding. Adri, hanging out with you is like a breath of fresh air; I routinely get nostalgic about our year as flatmates, and will never forget our trip to Croatia. Sara, your fearlessness in tackling problems has inspired me from the start, and your can-do attitude is a beacon for those around you. Sophie, your enthusiasm and passion for all the things you do is uplifting. You guys are brilliant and real rock stars.

I am also thankful for the Princeton Women in Physics group, particularly the past and present leadership. Laura Chang and Mallika Randeria paved the way for us in WiP, and Laura Zhang was a trailblazer for the Women in Plasma Physics group. Sara Sussman and Gillian Kopp were the best co-organizers I could ask for. I am also indebted to the current torch-bearers: Sophie Dvali, Emily Osborne, Adriana Dropulic, Pearl Thijssen, and Lindsay Smith. When I joined as a first-year graduate student, I was also blown away by the dedication of the students who established the Undergraduate Women in Physics (UWiP), and later TiCuP (Towards an Inclusive Community of Undergraduate Physicists). The amount of work and labor that goes into these groups cannot be overstated. These groups helped me feel like I truly belonged in Jadwin and at Princeton.

I would also like to voice my appreciation for the people I met during summers at CERN: Nicole, who picked me up the first time I ever landed at the Geneva airport, as well as Nico, Elise, Lucy, and Jorge, with whom I've had many great conversations. There are also friends from college who have been great to stay in touch with as we explore ``life after Tech": Michelle, Kenny, and many others. Wherever you guys go, I think the future will be bright.

Moving to CERN for the last two years of the PhD was super memorable and rewarding. A warm thank you to the crew here: Amy, Dan, Daniel, Christian, David, and my flatmates Gillian and Bennett, for the support, bonding over relocating to a new country, the climbing sessions, the weekend outings, going to the lake, and generally being down for various hijinks. I think some of those hikes will live rent-free in my head forever (at least two of them involved snow- so maybe that says something). Thanks for patiently listening to me wail about being in ``thesis jail" as I wrote this dissertation in a little over two months.

Thank you, Tyler, who says he ``didn't really do anything" for my PhD, but I would like to say that you helped me keep going many times over the years. You're the inspiration.

There are not quite enough words to express my gratitude for my parents, whose unconditional support has meant everything. I couldn't have done it without you.
