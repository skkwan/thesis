As the modeling of the CMS detector and underlying physics processes are imperfect, scale factors and corrections are applied to Monte Carlo events and in some cases, the embedded samples, to improve agreement with measurements in data. For the standard reconstruction physics objects in this work, central recommendations provided by dedicated physics object groups (POGs) in CMS are used.

Uncertainties in scale factors and corrections are also sources of systematic errors in the analysis.


\section{Lepton energy scales}

Lepton energy scales are applied to the $p_{T}$ and mass of the relevant objects ($\tau_{h}$, muon, and electron). Systematic uncertainties in the lepton energy scales can shift the $p_{T}$ of the leptons up or down, causing a migration of number of events that pass the offline $p_{T}$ thresholds described in the previous section.

The $\tau_{h}$ energy scale is binned in the four reconstructed decay topologies of the $\tau_{h}$ \cite{twiki_TAU_POG_tauidrecommendationforrun2}. When applying the energy scale to the $\tau_{h}$, the 4-momentum of the missing transverse energy (MET) is adjusted such that the total 4-momenta of the $\tau_{h}$ and the MET remains unchanged. 

Measurements of the muon energy scale have shown a relative $p_{T}$ resolution range of 1\% to 1.5\% for muons in the barrel and less than 6\% in the endcaps, which is reflected in the systematical uncertainties of the muon energy scale \cite{twiki_MUON_POG_recommendation}. They are applied to genuine $\mu$ from $\tau$ decays.

Corrections to the electron energy scale are applied to genuine $e$ from $\tau$ decays, and are binned in regions of $\eta$ \cite{twiki_Electron_POG_recommendation}.

\section{$\tau_{h}$ identification efficiency}
The $\tau_{h}$ identification efficiency can differ in data and MC \cite{twiki_TAU_POG_tauidrecommendationforrun2}. Recommended corrections are provided by the Tau POG. The identification efficiency is measured in $Z \rightarrow \tau\tau$ events in the $\mu\tau_{h}$ final state, and is binned in $p_{T}$ due to clear $p_{T}$ dependence of the DeepTau ID. 

\section{$e \rightarrow \tau_{h}$ and $\mu \rightarrow \tau_{h}$ misidentification efficiency}
Corrections on identification efficiencies are applied to genuine electrons and muons misidentified as $\tau$s to account for differences in data and MC. The specific values depend on the vs. electron and vs. muon discriminator working points used. For misidentified $e \rightarrow \tau_{h}$, the scale factors are split into barrel and endcap regions, dictated by the ECAL detector geometry. For misidentified $\mu \rightarrow \tau_{h}$, the scale factors are split into different $|\eta|$ regions, determined by the CMS muon and tracker detector geometries.


\section{Electron and muon ID, isolation, and tracking}


\section{MET recoil corrections}


\section{Drell-Yan corrections}

\section{Pileup reweighing}


\section{Trigger efficiencies}

\section{Top $p_{T}$ spectrum reweighing}


\section{B-tagging efficiency}

\section{Jet energy resolution and jet energy smearing}