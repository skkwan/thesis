As the modeling of the CMS detector and underlying physics processes are imperfect, scale factors and corrections are applied to Monte Carlo events and in some cases, the embedded samples, to improve agreement with measurements in data. For the standard reconstruction physics objects in this work, central recommendations provided by dedicated physics object groups (POGs) in CMS are used.

Uncertainties in scale factors and corrections are also sources of systematic errors in the analysis, detailed in Chapter \ref{chapter:ch-12:systematic-uncertainties}. Systematic uncertainties in the tau, muon, and electron energy scales can shift the $p_{T}$ of the leptons up or down, causing a migration of number of events that pass the offline $p_{T}$ thresholds described in the previous section.

\section{Tau energy scale}
\label{sec:tau_energy_scale}

An energy scale is applied to the transverse momentum $p_{T}$ and mass of the hadronic tau $\tau_{h}$ in the $\mu\tau_{h}$ and $e\tau_{h}$ channels, to correct for a deviation of the average reconstructed $\tau_{h}$ energy from the generator-level energy of the visible $\tau_{h}$ decay products. These correction factors are derived centrally \cite{CMS-TAU-16-003}, by fitting to events in $e\tau_{h}$ and $\mu\tau_{h}$ final states in $Z/\gamma^*$ events separately for the $h^\pm$, $h^\pm \pi^0$, and $h^\pm h^\mp h^\pm$ decays. The values used are shown in Table \ref{table:tau-ES}.

When applying the energy scale to the $\tau_{h}$, the 4-momentum of the missing transverse energy (MET) is adjusted such that the total 4-momenta of the $\tau_{h}$ and the MET remains unchanged \cite{twiki_TAU_POG_tauidrecommendationforrun2}.

\begin{table}[ht]
    \centering
    \begin{tabular}{|c|c|c|c|c|}
    \hline
    \multicolumn{5}{|c|}{Tau energy scale factor}                                   \\ \hline
    \hline
    Decay mode      & 2018              & 2017              & 2016 pre-VFP      & 2016 post-VFP     \\ \hline
    0               & 0.991 $\pm$ 0.008 & 0.986 $\pm$ 0.009 & 0.987 $\pm$ 0.01  & 0.993 $\pm$ 0.009 \\
    1               & 1.004 $\pm$ 0.006 & 0.999 $\pm$ 0.006 & 0.998 $\pm$ 0.006 & 0.991 $\pm$ 0.007 \\
    10              & 0.998 $\pm$ 0.007 & 0.999 $\pm$ 0.007 & 0.984 $\pm$ 0.008 & 1.001 $\pm$ 0.007 \\
    11              & 1.004 $\pm$ 0.009 & 0.996 $\pm$ 0.01  & 0.999 $\pm$ 0.011 & 0.997 $\pm$ 0.016 \\ \hline
    \end{tabular}
    \caption{Energy scales applied to genuine hadronic tau decays $\tau_{h}$ by data-taking year/era and decay mode, along with systematic errors.}
    \label{table:tau-ES}
\end{table}

\section{Muon energy scale}
\label{sec:muon_energy_scale}

An energy scale is applied to the $p_{T}$ and mass of genuine muons from $\tau$ decays in the $e\mu$ and $\mu\tau_{h}$ channels \cite{twiki_MUON_POG_recommendation}. The applied values are the same for MC and embedded samples and are shown in Table \ref{table:muon-ES}. Following the SM $H \rightarrow \tau\tau$ analysis, Rochester corrections are not applied, and instead prescriptions from \cite{twiki_MUO_simplified_ES} are followed.


\begin{table}[ht]
    \centering
    \begin{tabular}{|c|c|}
    \hline
    \multicolumn{2}{|c|}{Muon energy scale factor}      \\ \hline
    \hline
    Eta range                & Value for all years \\ \hline
    $|\eta| \in [0.0, 1.2)$  & 1.0 $\pm$ 0.004 \\
    $|\eta| \in [1.2, 2.1)$  & 1.0 $\pm$ 0.009 \\
    $|\eta| \in [2.1, 2.4)$  & 1.0 $\pm$ 0.027 \\
    \hline
    \end{tabular}
    \caption[Energy scales and systematic errors applied to genuine muons.]{Energy scales and systematic errors applied to genuine muons. The values are the same for MC and embedded for all years \cite{twiki_HiggsToTauTauWorkingLegacyRun2} \cite{twiki_MUO_simplified_ES}.}
    \label{table:muon-ES}
\end{table}


\section{Electron energy scale}
\label{sec:electron_energy_scale}

Corrections to the electron energy scale are applied to genuine $e$ from $\tau$ decays, and are binned in two dimensions by electron $p_{T}$ and $\eta$ for barrel vs. endcap \cite{twiki_Electron_POG_recommendation}. The scale factors are binned in $p_{T}$ and $\eta$ for MC samples: e.g. values for 2018 are shown in Fig. \ref{fig:egamma-POG-UL-egamma-scale-factors} from \cite{twiki_Electron_UL_2016_2017_2018}. For embedded samples the electron energy scale is taken as only binned in $\eta$ (Table \ref{table:ele-ES-embedded}).

% https://twiki.cern.ch/twiki/pub/CMS/EgammaUL2016To2018/egammaEffi.txt_Ele_wp90noiso_egammaPlots.pdf
\begin{figure}[h]
    \centering
    \includegraphics[width=15cm]{figures/ch-8-scale-factors-and-corrections/egamma-POG-UL-egamma-scale-factors.png}
    \caption[Electron/photon energy scale factors and uncertainties for 2018.]{Electron/photon energy scale factors (\textit{left}) and corresponding uncertainties (\textit{right}) binned in the electron $\eta$ and $p_{T}$, for the data-taking year 2018 \cite{twiki_Electron_UL_2016_2017_2018}.} 
    \label{fig:egamma-POG-UL-egamma-scale-factors}
\end{figure}


\begin{table}[h]
    \centering
    \begin{tabular}{|c|c|c|c|}
    \hline
    \multicolumn{4}{|c|}{Electron energy scale factor for embedded samples}                                   \\ \hline
    \hline
    Eta range                   & 2018               & 2017               & 2016     \\ \hline
    $|\eta| \in [0.0, 1.479)$   & 0.973 $\pm$ 0.005  & 0.986 $\pm$ 0.009  & 0.9976 $\pm$ 0.0050 \\
    $|\eta| \in [1.479, 2.4)$   & 0.980 $\pm$ 0.0125 & 0.887 $\pm$ 0.0125 & 0.993 $\pm$ 0.0125 \\ \hline
    \end{tabular}
    \caption[Energy scales and systematic errors applied to electrons in embedded samples by data-taking year/era.]{Energy scales and systematic errors applied to electrons in embedded samples, binned in the electron $\eta$, by data-taking year \cite{twiki_embedded_preUL_2016} \cite{twiki_embedded_preUL_2017} \cite{twiki_embedded_preUL_2018}.}
    \label{table:ele-ES-embedded}
\end{table}

\section{$\tau_{h}$ identification efficiency}
\label{sec:tauh_id_efficiency}

The $\tau_{h}$ identification efficiency can differ in data and MC \cite{twiki_TAU_POG_tauidrecommendationforrun2}. Recommended corrections are provided by the Tau POG, and we use the medium DeepTau vs. jet working point values. The identification efficiency is measured in $Z \rightarrow \tau\tau$ events in the $\mu\tau_{h}$ final state, and is binned in $p_{T}$ due to clear $p_{T}$ dependence of the DeepTau ID. 


\begin{table}[h]
    \centering
    \begin{tabular}{|c|c|c|c|c|c|c|}
    \hline
    \multicolumn{7}{|c|}{Tau ID efficiency for DeepTau Medium vs. jet WP in 2018}                                   \\ \hline
    \hline
    $p_{T}$ (GeV)  & $<20$  & $(20, 25]$ & $(25, 30]$ & $(30, 35]$ & $(35, 40]$ & $(40, 500] $   \\ \hline
    Central value  & 0      & 0.945      & 0.946      & 0.916      & 0.921      & 1.005 \\
    Up value       & 0      & 1.001      & 0.981      & 0.946      & 0.950      & 1.035 \\
    Down value     & 0      & 0.888      & 0.981      & 0.883      & 0.893      & 0.953 \\ \hline
    \end{tabular}
    \caption[Tau ID efficiency for the DeepTau vs. jet medium working point, with central, up, and down values for 2018, binned in the tau $p_{T}$.]{Tau ID efficiency for the DeepTau vs. jet medium working point, with central, up, and down values for 2018, binned in the tau $p_{T}$ \cite{twiki_TAU_POG_tauidrecommendationforrun2}.}
    \label{table:tauIDeff_deepTau_vs_jet_medium_WP}
\end{table}


\section{Trigger efficiencies}

Scale factors are applied to correct for differences in trigger efficiencies between MC and embedded vs. data, with values taken from tools provided by the Standard Model $H \rightarrow \tau\tau$ working group which uses the same trigger paths \cite{twiki_HiggsToTauTauWorkingLegacyRun2}. In the following sections we review relevant trigger efficiencies in data, which form the basis of the trigger efficiency corrections applied to MC and embedded.

\subsection{Tau trigger efficiencies}
The efficiencies in data of the single-$\tau_{h}$ leg in $\mu\tau_{h}$, $e\tau_{h}$, and di-$\tau_{h}$ triggers is computed centrally per using a Tag and Probe (TnP) method \cite{CMS-DP-2019-012} which is outlined here. In this method, $Z \rightarrow \tau\tau \rightarrow \mu\tau_{h}$ are selected in data and a Drell-Yan simulated sample ($Z \rightarrow \ell\ell, \ell = e, \mu, \tau_{h}$) with high purity. Cuts are applied to reject events not in this final state, e.g. suppressing $Z \rightarrow \mu\mu$ by vetoing events with a single loose ID muon. An isolated muon candidate (the tag) with online $p_{T} > 27$ GeV and $|\eta| < 2.1$ is identified and matched to an offline $\mu$. An offline $\tau_{h}$ candidate (the probe) is selected, which is separated from the tag $\mu$, and has $p_{T} > 20$ GeV and $|\eta| < 2.1$. The probe $\tau_{h}$ must pass anti-muon and anti-electron discriminators to avoid fakes from muons and electrons, and must pass the medium MVA tau isolation to suppress fakes from QCD jets. The trigger efficiency in the TnP method is calculated as 
\begin{equation}
    \text{Efficiency} = \frac{\text{Number of events passing the TnP selection with fires the HLT path}}{\text{Number of events passing the TnP selection}}
\end{equation}


The efficiencies for the hadronic tau legs in the relevant channels of this analyses ($\mu\tau_{h}$ and $e\tau_{h}$) as a function of the offline tau $p_{T}$ and $\eta$, are shown for data taken in 2016, 2017, and 2018 in Figures \ref{fig:mutauEfficiencyPt_eachYear_mediumTauMVA_Data} and \ref{fig:etauEfficiencyPt_eachYear_mediumTauMVA_Data} \cite{CMS-DP-2019-012} \cite{twiki_Tau_Lepton_Run_2_trigger_performance}. In both figures, the different HLT thresholds and differences in the L1 seed result in higher efficiencies in 2016 and differences in shapes of the 2016 efficiencies compared to 2017 and 2018. The low pileup in 2016 also leads to higher efficiencies in that year.


\begin{figure}[h]
    \centering
    \begin{subfigure}{0.45\textwidth}
        \includegraphics[width=1.0\textwidth]{figures/ch-8-scale-factors-and-corrections/mutauEfficiencyPt_eachYear_mediumTauMVA_Data.png}
        \caption{$\tau_{h}$ efficiency from $\mu\tau_{h}$ trigger.}
        \label{fig:mutauEfficiencyPt_eachYear_mediumTauMVA_Data}
    \end{subfigure}
    \hfill
    \begin{subfigure}{0.45\textwidth}
        \includegraphics[width=1.0\textwidth]{figures/ch-8-scale-factors-and-corrections/etauEfficiencyPt_eachYear_mediumTauMVA_Data.png}
        \caption{$\tau_{h}$ efficiency from $e\tau_{h}$ trigger.}
        \label{fig:etauEfficiencyPt_eachYear_mediumTauMVA_Data}
    \end{subfigure}
    \caption[Hadronic tau leg efficiency of the cross-triggers for $\mu\tau_{h}$ (\textit{left}) and $e\tau_{h}$ (\textit{right}) triggers as a function of offline tau $p_{T}$ for 2016, 2017, and 2018.]{Hadronic tau leg efficiency of the cross-triggers for $\mu\tau_{h}$ (\textit{left}) and $e\tau_{h}$ (\textit{right}) triggers as a function of offline tau $p_{T}$ for the years 2016 (\textit{red}), 2017 (\textit{blue}) and 2018 (\textit{green}), from \cite{twiki_Tau_Lepton_Run_2_trigger_performance}. HLT $p_{T}$ thresholds and L1 seeds are indicated in the legends.} 
\end{figure}


\subsection{Single muon trigger efficiencies}
The efficiencies for the single isolated muon trigger with $p_{T} > 24$ GeV used in this analysis, is shown for the data-taking year 2018 in Fig. \ref{fig:single_muon_24GeV_efficiency_vs_pt} as a function of the muon $p_{T}$ and as a function of the muon $|\eta|$ in Fig. \ref{fig:single_muon_24GeV_efficiency_vs_eta} from \cite{CMS-DP-2018-034}. The data is split with respect to a HLT muon reconstruction update that was deployed on 15/05/2018. A small asymmetry in efficiencies between negative and positive $\eta$ in Fig. \ref{fig:single_muon_24GeV_efficiency_vs_eta} is due to disabled muon chambers (CSCs). The efficiencies shown are estimated using a Tag and Probe method using $Z\rightarrow \mu\mu$ events, with the tag being an offline muon with $p_{T} > 29$ GeV and $|\eta| < 2.4$ passing a tight ID criteria, and the probe is an online (L1) trigger object with $\Delta R < 0.3$ and passing tight ID and Particle Flow based isolation requirements with $p_{T} > 26$ GeV.

\begin{figure}[h]
    \centering
    \begin{subfigure}{0.45\textwidth}
        \includegraphics[width=1.0\textwidth]{figures/ch-8-scale-factors-and-corrections/singleMuon_isolated_efficiency_vs_pt}
        \caption{Muon efficiency vs $p_{T}$ for SingleMuon.}
        \label{fig:single_muon_24GeV_efficiency_vs_pt}
    \end{subfigure}
    \hfill
    \begin{subfigure}{0.45\textwidth}
        \includegraphics[width=1.0\textwidth]{figures/ch-8-scale-factors-and-corrections/singleMuon_isolated_efficiency_vs_eta}
        \caption{Muon efficiency vs $|\eta|$ for SingleMuon.}
        \label{fig:single_muon_24GeV_efficiency_vs_eta}
    \end{subfigure}
    \caption[Trigger efficiencies in data (\textit{top panels}) and ratio of efficiencies after/before a HLT muon reconstruction update (\textit{bottom panels}) for the muon in the isolated single muon trigger with threshold $p_{T} > 24$ GeV in the data-taking year 2018, as functions of the muon $p_{T}$ (\textit{left}) and muon $|\eta|$ (\textit{right}).]{Trigger efficiencies in data (\textit{top panels}) and ratio of efficiencies after/before a HLT muon reconstruction update (\textit{bottom panels}) for the muon in the isolated single muon trigger with threshold $p_{T} > 24$ GeV in the data-taking year 2018, as functions of the muon $p_{T}$ (\textit{left}) and muon $|\eta|$ (\textit{right}). Only statistical errors are shown \cite{CMS-DP-2018-034}.} 
\end{figure}

\subsection{Single electron trigger efficiencies}

The efficiencies in data, and the ratio between data and MC, of the single electron HLT trigger with $p_{T}$ threshold 32 GeV used in this analysis are shown for 2018, as a function of the electron $p_{T}$ in Fig. \ref{fig:single_ele_32GeV_efficiency_vs_pt} and of the electron $|\eta|$ in Fig. \ref{fig:single_ele_32GeV_efficiency_vs_eta}, from \cite{CMS-DP-2020-016}. In the Tag and Probe method used for the 2018 dataset, the tag is an offline reconstructed electron with $|\eta| \leq 2.1$ and not in the barrel and endcap overlap region, with $p_{T} > 35$ GeV with tight isolation and shower shape requirements, firing the tag trigger. The probe is an offline reconstructed electron with $|\eta| \leq 2.5$ with $E_T^\text{ECAL} > 5$ GeV with no extra identification criteria \cite{CMS-DP-2020-016}. 

The disagreement between data and MC, particularly at low transverse momentum, is in part due to detector effects that are difficult to simulate, such as crystal transparency losses in the ECAL and the evolution of dead regions in the pixel tracker \cite{CMS-DP-2020-016}.

\begin{figure}[h]
    \centering
    \begin{subfigure}{0.45\textwidth}
        \includegraphics[width=1.0\textwidth]{figures/ch-8-scale-factors-and-corrections/electron_Ele32_WPTight_Gsf_efficiency_vsPt}
        \caption{Electron efficiency vs $p_{T}$ for single electron.}
        \label{fig:single_ele_32GeV_efficiency_vs_pt}
    \end{subfigure}
    \hfill
    \begin{subfigure}{0.45\textwidth}
        \includegraphics[width=1.0\textwidth]{figures/ch-8-scale-factors-and-corrections/electron_Ele32_WPTight_Gsf_efficiency_vsEta}
        \caption{Electron efficiency vs $|\eta|$ for single electron.}
        \label{fig:single_ele_32GeV_efficiency_vs_eta}
    \end{subfigure}
    \caption[Trigger efficiencies in data and the data/MC ratio for the electron in the single electron trigger with threshold $p_{T} > 32$ GeV in the data-taking year 2018, as functions of the electron $p_{T}$ (\textit{left}) and electron $|\eta|$ (\textit{right}).]{Trigger efficiencies in data, and the data/MC ratio for the electron in the single electron trigger with threshold $p_{T} > 32$ GeV in the data-taking year 2018, as functions of the electron $p_{T}$ (\textit{left}) and electron $|\eta|$ (\textit{right}) \cite{CMS-DP-2020-016}. In the plot vs. $p_{T}$, the region 1.442 $\leq |\eta| \leq$ 1.566 is not included as it corresponds to the transition between barrel and endcap parts of the ECAL.} 
\end{figure}


\subsection{$e\mu$ cross-trigger efficiencies}

The efficiencies of the electron and muons for the cross-trigger with leading muon used in the $e\mu$ channel are shown for data in 2016, 2017, and 2018 in Figures \ref{fig:ele_efficiency_vs_pT_emu} and \ref{fig:muon_efficiency_vs_eta_emu} \cite{CMS-DP-2019-025}. These efficiencies were measured centrally using a Tag and Probe in events with $Z$ to dileptons with the same flavour and opposite charge, where the tags are an isolated muon or electron, and the probe (offline) candidate is required to satisfy the same lepton selection as that of the tag candidate, be matched within $\Delta R < 0.1$ with a corresponding online trigger object, and also to pass the cross-trigger. The trigger efficiency is then:
\begin{equation}
    \text{Efficiency} = \frac{\text{Events passing lepton pair selections and probe passing trigger}}{\text{Events passing lepton pair selections}}
\end{equation}

\begin{figure}[h]
    \centering
    \begin{subfigure}{0.45\textwidth}
        \includegraphics[width=1.0\textwidth]{figures/ch-8-scale-factors-and-corrections/ele_efficiency_vs_pT_emu_HLT_Mu23_TrkIsoVVL_Ele12_CaloIdL_TrackIdL_IsoVL_DZ.png}
        \caption{Electron efficiency vs. $p_{T}$.}
        \label{fig:ele_efficiency_vs_pT_emu}
    \end{subfigure}
    \hfill
    \begin{subfigure}{0.45\textwidth}
        \includegraphics[width=1.0\textwidth]{figures/ch-8-scale-factors-and-corrections/muon_efficiency_vs_eta_emu_HLT_Mu23_TrkIsoVVL_Ele12_CaloIdL_TrackIdL_IsoVL_DZ.png}
        \caption{Muon efficiency vs. $\eta$.}
        \label{fig:muon_efficiency_vs_eta_emu}
    \end{subfigure}
    \caption[Efficiencies of the electron leg vs. $p_{T}$ (\textit{left}) and the muon log vs. $\eta$ (\textit{right}), for the HLT path with online thresholds of 12 GeV for the electron and 23 GeV for the muon, with the data-taking years 2016 through 2018 overlaid.]{Efficiencies of the electron leg vs. $p_{T}$ (\textit{left}) and the muon log vs. $\eta$ (\textit{right}), for the HLT path with online thresholds of 12 GeV for the electron and 23 GeV for the muon, for the data-taking years 2016 (\textit{black}), 2017 (\textit{red}), and 2018 (\textit{green}) \cite{CMS-DP-2019-025}.} 
\end{figure}


\section{Electrons and muons faking $\tau_{h}$: energy scales}

Energy scales for electrons misidentified as hadronic tau decays ($e$ faking $\tau_{h}$) are provided by the Tau POG, and were measured in the $e\tau_{h}$ channel with the visible invariant mass of the electron and hadronic tau system \cite{twiki_HiggsToTauTauWorkingLegacyRun2}. This energy scale is applied for $\tau_{h}$ with $p_{T} > 20$ GeV regardless of which DeepTau vs. electron working point was used. Values for 2018 are shown in Table \ref{table:electron-faking-tauh-FES-2018}.

% root -l TauFES_eta-dm_DeepTau2017v2p1VSe_2018ReReco.root 
\begin{table}[h]
    \centering
    \begin{tabular}{|c|c|}
    \hline
    \multicolumn{2}{|c|}{Electrons faking $\tau_{h}$ energy scale factor in 2018}      \\ \hline
    \hline
    Reconstructed decay mode of the fake $\tau_{h}$  & Central value and (up, down) shifts \\ \hline
    0   & 1.01362 (+0.00474, -0.00904) \\
    1   & 1.01945 (+0.01598, -0.01226) \\
    10  & 0.96903 (+0.0125, -0.03404) \\
    11 & 0.985 (+0.04309, -0.05499) \\ \hline
    \end{tabular}
    \caption[Energy scales and up/down systematic uncertainties applied to electrons misidentified as hadronic taus.]{Energy scales and up/down systematic uncertainties applied to electrons misidentified as hadronic taus for 2018, binned in decay mode of the fake $\tau_{h}$ \cite{twiki_HiggsToTauTauWorkingLegacyRun2}.}
    \label{table:electron-faking-tauh-FES-2018}
\end{table}

No nominal energy scale is applied for muons mis-reconstructed as $\tau_{h}$, and the uncertainty is treated as $\pm$ 1\% and uncorrelated in the reconstructed decay mode \cite{twiki_HiggsToTauTauWorkingLegacyRun2}. 

\section{Electrons and muons faking $\tau_{h}$: misidentification efficiencies}
Corrections on identification efficiencies are applied to genuine electrons and muons misidentified as $\tau$ to account for differences in data and MC.

The specific values depend on the vs. electron and vs. muon discriminator working points used. 
For misidentified $\mu \rightarrow \tau_{h}$, the scale factors are split into different $|\eta|$ regions, determined by the CMS muon and tracker detector geometries, as shown in Table \ref{table:tauIDeff_deepTau_vs_muon} for 2018 \cite{twiki_TAU_POG_tauidrecommendationforrun2}.


\begin{table}[h]
    \centering
    \begin{tabular}{|c|c|c|}
    \hline
    \multicolumn{3}{|c|}{Tau ID efficiency for DeepTau vs. muon WPs in 2018} \\ \hline
    \hline
    $|\eta|$  & Tight working point & VLoose working point \\ \hline
    (0.0, 0.2)     & 0.767 $\pm$ 0.127  & 0.954 $\pm$ 0.069  \\ \hline 
    (0.2, 0.6)     & 1.255 $\pm$ 0.258  & 1.009 $\pm$ 0.098  \\ \hline 
    (0.6, 1.0)     & 0.902 $\pm$ 0.203  & 1.029 $\pm$ 0.075 \\ \hline 
    (1.0, 1.45)    & 0.833 $\pm$ 0.415  & 0.928 $\pm$ 0.145\\ \hline
    (1.45, 2.0)    & 4.436 $\pm$ 0.814   & 5.000 $\pm$ 0.377 \\ \hline
    (2.0, 2.53)    & 1.000 $\pm$ 0.000         & 1.000 $\pm$ 0.000\\ \hline
    \end{tabular}
    \caption[Tau mis-identification efficiency for the DeepTau Tight and Very Loose (VLoose) working points vs. muons in 2018.]{Tau mis-identification efficiency for the DeepTau Tight and Very Loose (VLoose) working points vs. muons in 2018, binned in the muon $|\eta|$ \cite{twiki_TAU_POG_tauidrecommendationforrun2}.}
    \label{table:tauIDeff_deepTau_vs_muon}
\end{table}

For misidentified $e \rightarrow \tau_{h}$, the scale factors are split into barrel and endcap regions, dictated by the ECAL detector geometry, as shown in Table \ref{table:tauIDeff_deepTau_vs_electron} for 2018.

% root -l /Users/stephaniekwan/Dropbox/Princeton_G6/TauIDSFs/data/TauID_SF_eta_DeepTau2017v2p1VSe_2018ReReco.root 
\begin{table}[h]
    \centering
    \begin{tabular}{|c|c|c|}
    \hline
    \multicolumn{3}{|c|}{Tau ID efficiency for DeepTau vs. electron WPs in 2018} \\ \hline
    \hline
    $|\eta|$  & Tight working point & VLoose working point \\ \hline
    (0.0, 0.73)     & 1.47 $\pm$ 0.27  & 0.95 $\pm$ 0.07  \\ \hline 
    (0.73, 1.509)   & 1.509 $\pm$ 0.0  & 1.00 $\pm$ 0.0  \\ \hline 
    (1.509, 1.929)  & 1.929 $\pm$ 0.2  & 0.86 $\pm$ 0.1 \\ \hline 
    (1.929, 2.683)  & 2.683 $\pm$ 0.9  & 2.68 $\pm$ 0.0 \\ \hline
    \end{tabular}
    \caption[Tau mis-identification efficiency for the DeepTau Tight and Very Loose (VLoose) working points vs. electrons in 2018.]{Tau mis-identification efficiency for the DeepTau Tight and Very Loose (VLoose) working points vs. electrons in 2018, binned in the electron $|\eta|$ \cite{twiki_TAU_POG_tauidrecommendationforrun2}.}
    \label{table:tauIDeff_deepTau_vs_electron}
\end{table}


\section{Electron ID and tracking efficiency}
Scale factors are applied to MC to correct for differences between MC and data in the performance of electron identification (ID) and tracking.

Electron and photon identification, as discussed earlier, use variables with good signal vs. background discrimination power such as lateral shower shape and ratio of energy deposited in the HCAL to energy deposited in the ECAL at the position of the electron. The cut-based electron identification efficiencies in data and ratio of efficiencies in data to MC are shown in Fig. \ref{fig:electron_MVA_ID_efficiency} for the multivariate analysis (MVA) identification working point. 

The tracking efficiencies in data and the data/MC ratio are shown in Fig. \ref{fig:electron_GSF_tracking_efficiency} for the Gaussian-sum filter (GSF) tracking \cite{CMS-DP-2020-037}. 

\begin{figure}[h]
    \centering
    \begin{subfigure}{0.45\textwidth}
        \includegraphics[width=1.0\textwidth]{figures/ch-8-scale-factors-and-corrections/electron_MVA_90wp_identification_efficiency}
        \caption{Electron MVA ID.}
        \label{fig:electron_MVA_ID_efficiency}
    \end{subfigure}
    \hfill
    \begin{subfigure}{0.45\textwidth}
        \includegraphics[width=1.0\textwidth]{figures/ch-8-scale-factors-and-corrections/electron_gsf_tracking_efficiency}
        \caption{Electron GSF tracking.}
        \label{fig:electron_GSF_tracking_efficiency}
    \end{subfigure}
    \caption[Efficiencies in data (\textit{top panels}) and the ratio of efficiencies in data/MC (\textit{bottom panels}), for the electron multivariate analysis (MVA) identification (\textit{left}) and for the Gaussian-sum filter (GSF) tracking (\textit{right}).]{Efficiencies in data (\textit{top panels}) and the ratio of efficiencies in data/MC (\textit{bottom panels}), for the electron multivariate analysis (MVA) identification (\textit{left}) and for the Gaussian-sum filter (GSF) tracking (\textit{right}) \cite{CMS-DP-2020-037}. Error bars represent statistical and systematic uncertainties.} 
\end{figure}


\section{Muon ID, isolation, and tracking efficiencies}
Scale factors are applied to MC to correct for differences between MC and data in the performance of muon identification, isolation, and tracking, as detailed below.

The efficiencies for muon identification measured in 2015 data and MC simulation are shown in Figures \ref{fig:muon_looseID_efficiency} and \ref{fig:muon_tightID_efficiency} for the loose ID and tight ID respectively \cite{CMS-MUO-16-001}. The loose ID is chosen such that efficiency exceeds 99\% over the full $\eta$ range, and the data and simulation agree to within 1\%. The tight ID is chosen such that efficiency varies between 95\% and 99\% as a function of $\eta$, and the data and simulation agree to within 1-3\%. The muon identification working point used in this analysis is the medium ID, which has an efficiency of 98\% for all $\eta$ and an agreement within 1-2\% \cite{CMS-MUO-16-001}. 

\begin{figure}[h]
    \centering
    \begin{subfigure}{0.45\textwidth}
        \includegraphics[width=1.0\textwidth]{figures/ch-8-scale-factors-and-corrections/muon_efficiency_looseID}
        \caption{Muon efficiency isolation vs $p_{T}$.}
        \label{fig:muon_looseID_efficiency}
    \end{subfigure}
    \hfill
    \begin{subfigure}{0.45\textwidth}
        \includegraphics[width=1.0\textwidth]{figures/ch-8-scale-factors-and-corrections/muon_efficiency_tightID}
        \caption{Muon isolation efficiency vs. $|\eta|$.}
        \label{fig:muon_tightID_efficiency}
    \end{subfigure}
    \caption[Muon identification efficiencies in 2015 data and MC as a function of the muon $p_{T}$ for the loose ID (\textit{left}) and tight ID (\textit{right}) working points.]{Muon identification efficiencies in 2015 data and MC as a function of the muon $p_{T}$ for the loose ID (\textit{left}) and tight ID (\textit{right}) working points \cite{CMS-MUO-16-001}.} 
\end{figure}

The efficiencies in data for the muon isolation, as measured in Level-3 muons (muons in one of the final stages of reconstruction in the HLT), as a function of the muon $p_{T}$ and $|\eta|$ are shown in Figures \ref{fig:muon_isolation_efficiency_vsPt} and \ref{fig:muon_isolation_efficiency_vsEta} \cite{CMS-MUO-16-001}. The HLT muon reconstruction consists of two steps: Level-2 (L2), where the muon is reconstructed in the muon subdetectors only, and Level-3 (L3) which is a global fit of tracker and muon hits (i.e. the global muon reconstruction as described in Section \ref{section:ch-5-muon-reconstruction}) \cite{Verwilligen-proceedings-2016}.

\begin{figure}[h]
    \centering
    \begin{subfigure}{0.45\textwidth}
        \includegraphics[width=1.0\textwidth]{figures/ch-8-scale-factors-and-corrections/muon_efficiency_isolation_vsPt}
        \caption{Muon efficiency isolation vs $p_{T}$.}
        \label{fig:muon_isolation_efficiency_vsPt}
    \end{subfigure}
    \hfill
    \begin{subfigure}{0.45\textwidth}
        \includegraphics[width=1.0\textwidth]{figures/ch-8-scale-factors-and-corrections/muon_efficiency_isolation_vsEta}
        \caption{Muon isolation efficiency vs. $|\eta|$.}
        \label{fig:muon_isolation_efficiency_vsEta}
    \end{subfigure}
    \caption[Muon isolation efficiencies in Run-2 data as a function of the muon $p_{T}$ (\textit{left}) and $|\eta|$ (\textit{right}).]{Muon isolation efficiencies in Run-2 data with respect to Level-3 muons (one of the final stages of HLT muon reconstruction) as a function of the muon $p_{T}$ (\textit{left}) and $|\eta|$ (\textit{right}) \cite{CMS-MUO-16-001}.} 
\end{figure}

The muon tracking efficiencies as a function of $|\eta|$ for standalone muons (i.e. tracks from only the muon system, i.e. DT, CSC, and RPC, as discussed in Section \ref{section:ch-5-muon-reconstruction}), is shown for data and simulated Drell-Yan samples in Fig. \ref{fig:muon_tracking_efficiency} \cite{CMS-DP-2020-035}. 

\begin{figure}[h]
    \centering
    \includegraphics[width=8cm]{figures/ch-8-scale-factors-and-corrections/muon_tracking_efficiency}
    \caption[Muon tracking efficiencies as a function of $|\eta|$ for standalone muons in Run-2 data (\textit{black}) and Drell-Yan (\textit{blue}) MC simulation.]{Muon tracking efficiencies as a function of $|\eta|$ for standalone muons in Run-2 data (\textit{black}) and Drell-Yan MC simulation (\textit{blue}) \cite{CMS-DP-2020-035}. All Tracks refers to tracks which exploit the presence of muon candidates in the muon system to seed the track reconstruction in the inner tracker, in contrast to tracks that use tracker-only hits for seeding. Uncertainties shown are statistical.}
    \label{fig:muon_tracking_efficiency}
\end{figure}

    

\section{Recoil corrections}
\label{sec:ch-8-recoil-corrections}
In proton-proton collisions, W and Z bosons are predominantly produced through quark-antiquark annihilation. Higher-order processes can induce radiated quarks or gluons that recoil against the boson, imparting a non-zero transverse momentum to the boson \cite{2009-Tevatron-recoil-correction}. Recoil corrections accounting for this effect are applied to samples with W+jets, Z+jets, and Higgs bosons \cite{twiki_HiggsToTauTauWorkingLegacyRun2}. The corrections are performed on the vectorial difference between the measured missing transverse momentum and the total transverse momentum of neutrinos originating from the decay of the W, Z, or Higgs boson. This vector is projected onto the axes parallel and orthogonal to the boson $p_{T}$. This vector, and the resulting correction to use, is measured in $Z \rightarrow \mu\mu$ events, since these events have leptonic recoil that do not contain neutrinos, allowing the 4-vector of the Z boson to be be measured precisely. The corrections are binned in generator-level $p_{T}$ of the parent boson and also the number of jets in the event.

\section{Drell-Yan corrections}
The Z boson transverse momentum distribution disagrees between leading-order (LO) simulations and data in a $Z \rightarrow \mu\mu$ control region with at least one b-tag jet \cite{CMS-HIG-17-024}. Per-event weights derived by the 2016 data-only version of this analysis \cite{CMS-HIG-17-024} are applied to $Z \rightarrow \tau\tau / \ell \ell$ events, as a function of the generator-level Z boson $p_{T}$ to provide better matching of MC to data.

\section{Pileup reweighing}
Reweighing is performed to rescale MC events to account for differences between MC and data, in the distribution of the pileup (number of additional proton-proton interactions per bunch crossing). A tool for calculating the pileup reweighing for the MC samples used is provided centrally by the Luminosity POG \cite{twiki_LUMI_POG_recommendation}.

\section{Pre-firing corrections}
In 2016 and 2017 data-taking, a gradual timing shift of ECAL was not properly propagated to L1 trigger primitives (TPs), resulting in a large fraction of high $\eta$ TPs being incorrectly associated with the previous bunch crossing. L1 trigger rules prevent two consecutive bunch crossings from firing, causing events to be rejected if significant ECAL energy was deposited in $2.0 < |\eta| 3.0$. To account for this issue, MC simulations for 2016 and 2017 are corrected using an event-dependent weight. Embedded samples are not corrected \cite{CMS-HIG-19-010}.

\section{Top $p_{T}$ spectrum reweighing}
In Run-1 and Run-2 it was observed that the $p_{T}$ spectra of top quarks in $t\bar{t}$ data was significantly softer than those predicted by MC simulations \cite{twiki_Top_pt_reweighing}. Possible sources of this discrepancy are higher order QCD and/or electroweak corrections, and non-resonant production of $t\bar{t}$-like final states. To account for this, corrections derived from Run-2 data by the Top Physics Analysis Group (PAG) are applied to the $p_{T}$ of the top and anti-top quarks in MC simulations, computed as a function of their generator-level $p_{T}$ \cite{twiki_Top_pt_reweighing}.

\section{B-tagging efficiency}
In order to predict correct b-tagging discriminant distributions and event yields in data, the weight of selected MC events is reweighed according to recommendations by the BTV POG \cite{twiki_btag_SF_methods}. The reweighing depends on the jet $p_{T}$, $\eta$, and the b-tagging discriminant. In this method, there is no migration of events from one b-tag multiplicity bin to another.

\section{Jet energy resolution and jet energy smearing}
Calibration of jet energies, i.e. ensuring that the energy and momentum of the reconstructed jet matches that of the quark/gluon-initiated jet, is a challenging task due to time-dependent changes in the detector response and calibration and high pileup \cite{CMS-JME-13-004} \cite{proceedings-Agarwal:2022txa}. Jet calibration is done via jet energy corrections (JECs) applied to the $p_{T}$ of jets in MC samples, accounting successively for the effects of pileup, uniformity of the detector response, and residual data-simulation jet energy scale differences \cite{twiki_JetResolution_JEC}. Typical jet energy resolutions reported at $\sqrt{s} = 8$ TeV in the central rapidities are 15-20\% at 30 GeV and about 10\% at 100 GeV \cite{CMS-JME-13-004}. Jet energy corrections are also propagated to the missing transverse energy.

Measurements show that the jet energy resolution (JER) in data is worse than in simulation, and so the jets in MC need to be smeared to describe the data. JER corrections are applied after JEC on MC simulations, and adjust the width of the $p_{T}$ distribution based on pileup, jet size, and jet flavour \cite{twiki_JetResolution_JER}. Tools for applying JEC and JER are provided centrally by the JER Corrections group. 