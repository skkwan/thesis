\section{General procedure}
For the search for $h \rightarrow aa \rightarrow bb\tau\tau$, three final states of the $\tau\tau$ system are considered: $\mu\tau_{h}$, $e\tau_{h}$, and $e\mu$. The $\tau_{h}\tau_{h}$ final state is not considered because signal events in the $\tau_{h}\tau_{h}$ channel would typically produce hadronic taus with momenta below data-taking trigger thresholds.

In all three final states, events are required to have at least one b-tag jet passing the medium working point of the DeepFlavour tagger, with $p_{T} > 20$ GeV, and $|\eta| < 2.4$. A second b-tag jet is not required because such a requirement would reduce signal acceptance by 80\% compared to only requiring one b-tag jet.


Events in MC samples are sorted into one of the three $\tau\tau$ channels if they pass the following trigger requirements and requirements on the offline reconstructed objects in the event, first checking the HLT paths for the $\mu\tau_{h}$ channel, then $e\tau_{h}$, and finally $e\mu$. The two leading leptons (e.g. muon and hadronic tau for the $\mu\tau_{h}$ channel) that were determined to have originated from the $\tau\tau$ decay, are called the $\tau\tau$ ``legs'' and are respectively subscripted $1$ and $2$ in this work. For events in data and embedded samples, the HLT paths requirements for the corresponding channel are checked.

After sorting events by HLT paths and identifying the leading tau legs in the offline reconstructed objects, the $p_{T}$ of the offline objects is checked against the online trigger thresholds. Trigger matching is also performed, which checks the correspondence between each offline reconstructed object used in the analysis (e.g. a muon), and a trigger object in the HLT (e.g. a HLT muon). An offline object is considered to be matched, if it corresponds to a trigger object of the same object type, with $\Delta R < 0.5$. This matched trigger object is also required to pass the filter(s) of the HLT trigger.

Further cuts are made on the offline objects in each channel to obtain the signal region, or other data regions used to perform data-driven background estimations.

\section{Event selection in the $\mu\tau_{h}$ channel}

In all three years, a single muon trigger is used if the muon has sufficiently high $p_{T}$, otherwise a dilepton $\mu\tau_{h}$ cross-trigger is used (Tables \ref{table:trigger2016}, \ref{table:trigger2017}, and \ref{table:trigger2018}). For data taken in 2017-2018 (2016), the logical OR of the single muon triggers with online $p_{T}$ thresholds 24 and 27 (23) GeV is used, with the corresponding offline muon required to have with $p_{T}$ 1 GeV above the online threshold. For data taken in 2017-2018 (2016), a dilepton $\mu + \tau_{h}$ cross-trigger with $p_{T}$ thresholds of 20 (19) and 27 (20) GeV for the muon and tau respectively, is used. The $\tau_{h}$ is required to have $|\eta| < 2.3$ if the single trigger is fired, $|\eta| < 2.1$. 

The muon and $\tau_h$ are required to have opposite charge and be separated by $\Delta R > 0.4$. The muon is required to pass the medium identification working point, and have a relative isolation (computed in a cone size of $\Delta R = 0.4$) of less than 0.15. The muon is required to have $|\eta| < 2.4$, and the $\tau_{h}$ is required to have $|\eta| < 2.3$ unless a cross-trigger is required, in which case we require $|\eta| < 2.1$ as discussed above.


\section{Event selection in the $e\tau_{h}$ channel}

The HLT trigger paths for the $e\tau_{h}$ channel are summarized in Tables \ref{table:trigger2016}, \ref{table:trigger2017}, and \ref{table:trigger2018}. Similarly to the $\mu\tau_{h}$ channel, a single electron trigger is used if the electron has sufficiently high $p_{T}$ in 2018 and 2017. For data taken in 2018 (2017), the OR of the single electron triggers with online $p_{T}$ thresholds at 32 and 35 (27 and 32) GeV are used, with the corresponding offline electrons required to have $p_{T}$ greater than 33 (28) GeV. A $e + \tau_{h}$ cross-trigger is used for electrons with lower offline $p_{T}$ between 25 and 33 GeV (25 and 28 GeV). For the 2016 dataset, there is no cross trigger but only a single electron trigger with online $p_{T}$ threshold at 25 GeV, which is used if the offline electron has $p_{T}$ greater than 26 GeV.

The electron and $\tau_h$ are required to have opposite charge and be separated by $\Delta R > 0.4$. The electron is required to be within $|\eta| < 2.3$ when no cross trigger is used, and $|\eta| < 2.1$ when the cross trigger is fired. The electron is also required to have a relative isolation (cone size of $\Delta R = 0.4$) of less than 0.15. The $\tau_{h}$ is required to have $|\eta| < 2.3$ if no cross trigger is fired, and have $|\eta| < 2.1$ if the cross trigger is fired.

\section{Event selection $e\mu$ channel}


The HLT trigger paths for the $e\mu$ channel are summarized in Tables \ref{table:trigger2016}, \ref{table:trigger2017}, and \ref{table:trigger2018}. Events are selected with the logical OR of two $e+\mu$ cross triggers, where either the electron or muon can have larger $p_{T}$: (1) leading electron, where the electron has online $p_{T} > 23$ GeV and muon has online $p_{T} > 8$ GeV, or (2) leading muon, where electron has online $p_{T} > 12$ GeV and muon has online $p_{T}>23$ GeV.

The leading and sub-leading leptons are required to have an offline $p_{T}$ greater than 1 GeV above the online threshold (i.e. $p_{T} > 24$ GeV). If the sub-leading lepton is the electron, the offline $p_{T}$ threshold is 1 GeV above the online ($p_{T} > 13$ GeV), but if it is a muon, the offline $p_{T}$ threshold is required to be at least 5 GeV greater than the online threshold (i.e. $p_T > 13$ GeV). This is because of poor data and simulation agreement for low-$p_T$ muons with $p_T$ between 9 GeV and 13 GeV, and the higher probability of mis-identifying jets as muons at lower $p_{T}$. With no effect on the expected limits, the offline $p_{T}$ threshold for muons is raised to 13 GeV instead of 9 GeV, even though it may lead to loss in signal acceptance. 

The electron and muon are required to have opposite charge and be separated by $\Delta R > 0.3$ (note the decreased separation requirement compared to the other two channels). The electron is required to pass the MVA identification wokring point with 90\% efficiency, and to have a relative isolation (cone size of $\Delta R = 0.4$) less than 0.1. The muon is required to pass the medium identification working point, and to have a relative isolation (cone size of $\Delta R = 0.4$) less than 0.15. Both the electron and muon are required to be within $|\eta| < 2.4$.


\begin{table}[]
    \centering
    \begin{tabular}{ll}
    \hline
    \multicolumn{2}{|c|}{\footnotesize{2016 $\mu\tau_{h}$ trigger paths}}                                     \\ \hline
    \footnotesize{Notes}          & \footnotesize{HLT Path}                                           \\ \hline
                                  & \footnotesize{HLT\_IsoMu22\_v}                                    \\
                                  & \footnotesize{HLT\_IsoMu22\_eta2p1\_v}                            \\
                                  & \footnotesize{HLT\_IsoTkMu22\_v}                                  \\
                                  & \footnotesize{HLT\_IsoTkMu22\_eta2p1\_v}                          \\
                                  & \footnotesize{HLT\_IsoMu19\_eta2p1\_LooseIsoPFTau20\_v}           \\
                                  & \footnotesize{HLT\_IsoMu19\_eta2p1\_LooseIsoPFTau20\_SingleL1\_v} \\ \hline
    \multicolumn{2}{|c|}{\footnotesize{2016  $e\tau_{h}$ trigger paths}}                                         \\ \hline
    \footnotesize{Notes}          & \footnotesize{HLT Path}                                           \\ \hline
                                  & \footnotesize{HLT\_Ele25\_eta2p1\_WPTight\_Gsf\_v}                \\ \hline
    \multicolumn{2}{|c|}{\footnotesize{2016 $e\mu$ trigger paths}}                                            \\ \hline
    \footnotesize{Notes}           & \footnotesize{HLT Path}                                                     \\ \hline
    \footnotesize{runs B-F and MC} & \footnotesize{HLT\_Mu23\_TrkIsoVVL\_Ele12\_CaloIdL\_TrackIdL\_IsoVL\_v}     \\
    \footnotesize{runs B-F and MC} & \footnotesize{HLT\_Mu8\_TrkIsoVVL\_Ele23\_CaloIdL\_TrackIdL\_IsoVL\_v}      \\
    \footnotesize{runs G-H}        & \footnotesize{HLT\_Mu23\_TrkIsoVVL\_Ele12\_CaloIdL\_TrackIdL\_IsoVL\_DZ\_v} \\
    \footnotesize{runs G-H}        & \footnotesize{HLT\_Mu8\_TrkIsoVVL\_Ele23\_CaloIdL\_TrackIdL\_IsoVL\_DZ\_v} 
    \end{tabular}
    \caption{High-Level Trigger (HLT) paths used to select data and simulation events in 2016 for the three $\tau\tau$ channels.}
    \label{table:trigger2016}
\end{table}

    
\begin{table}[]
    \centering
    \begin{tabular}{ll}
    \hline  
    \multicolumn{2}{|c|}{\footnotesize{2017 $\mu\tau_{h}$ trigger paths}}                                     \\ \hline
    \footnotesize{Notes}         & \footnotesize{HLT Path}                                                           \\ \hline
                                 & \footnotesize{HLT\_IsoMu24\_v}                                                    \\
                                 & \footnotesize{HLT\_IsoMu27\_v}                                                    \\
                                 & \footnotesize{HLT\_IsoMu20\_eta2p1\_LooseChargedIso\_PFTau27\_eta2p1\_CrossL1\_v} \\ \hline
    \multicolumn{2}{|c|}{\footnotesize{2017  $e\tau_{h}$ trigger paths}}                                         \\ \hline
    \footnotesize{Notes}         & \footnotesize{HLT Path}                                                       \\ \hline
                                 & \footnotesize{HLT\_Ele32\_WPTight\_Gsf\_v}                                                    \\
                                 & \footnotesize{HLT\_Ele35\_WPTight\_Gsf\_v}                                                    \\
                                 & \footnotesize{HLT\_Ele24\_eta2p1\_WPTight\_Gsf\_Loose\_ChargedIsoPFTau30\_eta2p1\_CrossL1\_v} \\ \hline
    \multicolumn{2}{|c|}{\footnotesize{2017 $e\mu$ trigger paths}}                                            \\ \hline
    \footnotesize{Notes}         & \footnotesize{HLT Path}                                                        \\ \hline
                                 & \footnotesize{HLT\_Mu23\_TrkIsoVVL\_Ele12\_CaloIdL\_TrackIdL\_IsoVL\_DZ\_v}    \\
                                 & \footnotesize{HLT\_Mu8\_TrkIsoVVL\_Ele23\_CaloIdL\_TrackIdL\_IsoVL\_DZ\_v}                   
    \end{tabular}
    \caption{High-Level Trigger (HLT) paths used to select data and simulation events in 2017 for the three $\tau\tau$ channels.}
    \label{table:trigger2017}
\end{table}
    
    

\begin{table}[]
    \centering
    \begin{tabular}{ll}
    \hline
    \multicolumn{2}{|c|}{\footnotesize{2018 $\mu\tau_{h}$ trigger paths}}                                     \\ \hline
    \footnotesize{Notes}          & \footnotesize{HLT Path}                                                   \\ \hline
                                  & \footnotesize{HLT\_IsoMu24\_v}                                                           \\
                                  & \footnotesize{HLT\_IsoMu27\_v}                                                           \\
    \footnotesize{only data run $<$ 317509}      & \footnotesize{HLT\_IsoMu20\_eta2p1\_ (contd.)} \\
                                                 & \footnotesize{LooseChargedIsoPFTauHPS27\_eta2p1\_CrossL1\_v} \\
    \footnotesize{MC and data run $\geq$ 317509} & \footnotesize{HLT\_IsoMu20\_eta2p1\_ (contd.)} \\
                                                 & \footnotesize{LooseChargedIsoPFTauHPS27\_eta2p1\_TightID\_CrossL1\_v}     \\ \hline
    \multicolumn{2}{|c|}{\footnotesize{2018 $e\tau_{h}$ trigger paths}}                                       \\ \hline
    \footnotesize{Notes}          & \footnotesize{HLT Path}                                                   \\ \hline
                                  & \footnotesize{HLT\_Ele32\_WPTight\_Gsf\_v}                                               \\
                                  & \footnotesize{HLT\_Ele35\_WPTight\_Gsf\_v}                                               \\
    \footnotesize{only data run $<$ 317509}      & \footnotesize{HLT\_Ele24\_eta2p1\_WPTight\_Gsf\_ (contd.)} \\
                                                 & \footnotesize{LooseChargedIsoPFTauHPS30\_eta2p1\_CrossL1\_v}          \\
    \footnotesize{MC and data run $\geq$ 317509} & \footnotesize{HLT\_Ele24\_eta2p1\_WPTight\_Gsf\_ (contd.)} \\
                                                 & \footnotesize{LooseChargedIsoPFTauHPS30\_eta2p1\_TightID\_CrossL1\_v} \\ \hline
    \multicolumn{2}{|c|}{\footnotesize{2018 $e\mu$ trigger paths}}                                            \\ \hline
    \footnotesize{Notes}          & \footnotesize{HLT Path}                                                   \\ \hline
                                  & \footnotesize{HLT\_Mu23\_TrkIsoVVL\_Ele12\_CaloIdL\_TrackIdL\_IsoVL\_DZ\_v}              \\
                                  & \footnotesize{HLT\_Mu8\_TrkIsoVVL\_Ele23\_CaloIdL\_TrackIdL\_IsoVL\_DZ\_v}                             
    \end{tabular}
    \caption{High-Level Trigger (HLT) paths used to select data and simulation events in 2018 for the three $\tau\tau$ channels. In 2018 a HLT trigger path using the hadron plus strips (HPS) tau reconstruction algorithm became available.}
    \label{table:trigger2018}
\end{table}

