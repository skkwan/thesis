For the search for $h \rightarrow aa \rightarrow 2b2\tau$, three final states of the $\tau\tau$ system are considered: $\mu\tau_{h}$, $e\tau_{h}$, and $e\mu$. The $\tau_{h}\tau_{h}$ final state is not considered because signal events in the $\tau_{h}\tau_{h}$ channel would typically produce hadronic taus with momenta below data-taking trigger thresholds.

In all three final states, events are required to pass at least one b-tag jet passing the medium working point of the DeepFlavour tagger, with $p_{T} > 20$ GeV, and $|\eta| < 2.4$. A second b-tag jet is not required because such a requirement would reduce signal acceptance by 80\% compared to only requiring one b-tag jet.

Events in data, MC samples, and Embedded samples are sorted into one of the three $\tau\tau$ channels if they pass the following trigger requirements and requirements on the offline reconstructed objects in the event. The two leading leptons (e.g. muon and hadronic tau for the $\mu\tau_{h}$ channel) determined to have originated from the $\tau\tau$ decay, are called the leading ``legs'' and are respectively subscripted $1$ and $2$ in this work.


\section{$\mu\tau_{h}$ channel}

A single muon trigger is used if the muon has sufficiently high $p_{T}$, otherwise a dilepton $\mu\tau_{h}$ cross-trigger is used. For data taken in 2017-2018 (2016), the logical OR of the single muon triggers with online $p_{T}$ thresholds 24 and 27 (23) GeV is used, with the corresponding offline muon required to have with $p_{T}$ 1 GeV above the online threshold. For data taken in 2017-2018 (2016), a dilepton $\mu + \tau_{h}$ cross-trigger with $p_{T}$ thresholds of 20 (19) and 27 (20) GeV for the muon and tau respectively, is used. The $\tau_{h}$ is required to have $|\eta| < 2.3$ if the single trigger is fired, $|\eta| < 2.1$. 

\section{$e\tau_{h}$ channel}

Similarly to the $\mu\tau_{h}$ channel, a single electron trigger is used if the electron has sufficiently high $p_{T}$ in 2018 and 2017. For data taken in 2018 (2017), the OR of the single electron triggers with online $p_{T}$ thresholds at 32 and 35 (27 and 32) GeV are used, with the corresponding offline electrons required to have $p_{T}$ greater than 33 (28) GeV. A $e + \tau_{h}$ cross-trigger is used for electrons with lower offline $p_{T}$ between 25 and 33 GeV (25 and 28 GeV). For the 2016 dataset, there is no cross trigger but only a single electron trigger with online $p_{T}$ threshold at 25 GeV, which is used if the offline electron has $p_{T}$ greater than 26 GeV.

\section{$e\mu$ channel}

In the $e\mu$ channel, events are selected with the logical OR of two $e+\mu$ cross triggers: electron with online $p_{T} > 23$ GeV and muon with online $p_{T} > 8$ GeV, or electron with online $p_{T} > 12$ GeV and muon with online $p_{T}>23$ GeV. The corresponding offline objects are required to have offline $p_{T}$ several GeV greater than than the online thresholds.