With the discovery of the Higgs boson with mass 125\GeV at the Large Hadron Collider (LHC) in 2012, the Compact Muon Solenoid (CMS) experimental physics program has evolved to include the precise characterization of the Standard Model 125\GeV Higgs boson and searches for additional particles in an extended Higgs sector. Properties of an extended Higgs sector remain loosely constrained by current measurements, making direct searches for exotic Higgs decays a powerful probe of new physics. The decay of the 125\GeV Higgs into two light neutral scalar particles ($h\rightarrow aa$) is allowed in extensions of the Standard Model, such as Two Higgs Doublet Models extended with a scalar singlet (2HDM+S). We present a search at CMS for exotic decays of the 125\GeV Higgs to two light neutral scalars, which decay to two bottom quarks and two tau leptons ($h\rightarrow aa \rightarrow bb\tau\tau$). This analysis is combined with a similar search in a final state with two bottom quarks and two muons. The results from the $bb\tau\tau$ analysis and the combined analyses are interpreted in 2HDM+S scenarios. In the Two Real Singlet Model (TRSM), the two light scalars can have unequal mass ($h\rightarrow a_1 a_2$). This decay has not been searched for to date at CMS. We present ongoing work on a search for $h\rightarrow a_1 a_2$, where the $a_2$ decays into two $a_1$, resulting in four bottom quarks and two tau leptons, in the $\mu\tau_{h}$ channel of the $\tau\tau$ decay. There remain rare signatures from Standard Model and beyond-Standard Model physics that are challenging to probe with current datasets. To improve the discovery potential of the LHC, the High-Luminosity LHC (HL-LHC) is scheduled to deliver a dataset around ten times larger than the combined dataset of LHC Runs 1-3, and will increase the number of simultaneous proton-proton collisions (pile-up) by a factor of five to seven. The hardware-based CMS Level-1 (L1) Trigger will be upgraded in order to filter data in these intense pile-up conditions. This thesis also presents an L1 Trigger algorithm that will use information with higher spatial granularity to reconstruct and identify electrons and photons in the barrel calorimeter.
