\documentclass{article}

% Language setting
% Replace `english' with e.g. `spanish' to change the document language
\usepackage[english]{babel}

% Set page size and margins
% Replace `letterpaper' with `a4paper' for UK/EU standard size
\usepackage[letterpaper,top=2cm,bottom=2cm,left=3cm,right=3cm,marginparwidth=1.75cm]{geometry}

% Useful packages
\usepackage{amsmath}
\usepackage{amssymb}
\usepackage{graphicx}
\usepackage[colorlinks=true, allcolors=blue]{hyperref}

\begin{document}

\section{CERN, the LHC, and the CMS Experiment}

\subsection{CERN and the Large Hadron Collider}
% [CITE] https://cds.cern.ch/record/782076
The European Council for Nuclear Research (in French \textit{Conseil Europ\'{e}en pour la Recherche Nucl\'{e}aire}), also known as CERN, is the site of an accelerator complex hosting the Large Hadron Collider (LHC). The LHC consists of a 27-kilometer ring of superconducting magnets with accelerating structures to boost the energy of particles, which collide at a center-of-mass energy of up to 14 TeV. The beams inside the LHC are made to collide at four locations around the accelerator ring, at the locations of four particle detectors: ATLAS, CMS, ALICE, and LHCb.

The number of events generated per second at the LHC collisions is given by $N_{event} = \mathcal{L} \sigma_{event}$, where $\sigma_{event}$ is the cross-section for the event under study, and $\mathcal{L}$ the machine luminosity. The machine luminosity depends only on the beam parameters, and can be written for a Gaussian beam distribution as:

\begin{equation}
    \mathcal{L} = \frac{N_b^2 n_b f_{rev} \gamma_r}{4\pi \epsilon_n \beta^*} F
\end{equation}
where $N_b$ is the number of particles per bunch, $n_b$ the number of bunches per beam, $f_{rev}$ the revolution frequency, $\gamma_r$ the relativistic gamma factor, $\epsilon_n$ the normalized transverse beam emittance, $\beta^*$ the beta function at the collision point, and $F$ the geometric luminosity reduction factor due to the crossing angle at the interaction points. Luminosity is measured in units of cm$^{-2}$ s$^{-1}$. Thus the exploration of rare events in the LHC collisions requires both high beam energies and high beam intensities.

\subsection{The CMS Detector}
% cite https://iopscience.iop.org/article/10.1088/1748-0221/3/08/S08004/pdf
The Compact Muon Solenoid (CMS) experiment was conceived to study proton-proton and lead-lead collisions at a center-of-mass energy of 14 TeV (5.5 TeV nucleon-nucleon) and at luminosities up to $10^{34}$ cm$^{-2}$ s$^{-1}$ ($10^{27}$ cm$^{-2}$ s$^{-1}$). At the center of the CMS detector, a high-magnetic-field superconducting solenoid surrounds a silicon pixel and strip tracker, a lead-tungstate scintillating-crystals electromagnetic calorimeter (ECAL), and a brass-scintillator sampling hadron calorimeter (HCAL). The iron yoke of the flux-return houses four stations of gas-ionization chamber muon detectors. The collision data is recorded with the use of the Level-1 (L1) trigger, high-level trigger (HLT), and data acquisition systems ensuring high efficiency in selecting physics events of interest. A detailed description of the CMS detector, along with definitions of coordinate system and relevant kinematic variables, can be found in [CITE].

\subsection{Sub-detectors of CMS}

\end{document}