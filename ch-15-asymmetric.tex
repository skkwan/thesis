In this chapter, progress towards a search for exotic Higgs decays to two light scalars with unequal mass ($h \rightarrow a_1 a_2$) final states with bottom quarks and $\tau$ leptons is presented. To date, the asymmetric decay $h_{125} \rightarrow a_1 a_2$ has not been directly searched for at the CMS experiment.

\section{Signal mass points}
As discussed in Section \ref{section:theory-TRSM}, $h \rightarrow a_1 a_2$ can result in a ``cascade" decay if one of the scalars, $a_2$ is sufficiently heavy ($m_{a_2} > 2m_{a_1}$). The ``non-cascade" case is where the light scalars decay directly to Standard Model particles. 

The mass points $(m_{a_1}, m_{a_2})$ studied here are:
\begin{itemize}
    \item \textit{Cascade mass points:} (15, 30), (15, 40), (15, 50), (15, 60), (15, 70), (15, 80), (15, 90), (15, 100), (15, 110), (20, 40), (20, 50), (20, 60), (20, 70), (20, 80), (20, 90), (20, 100), (30, 60), (30, 70), (30, 80), and (30, 90) GeV
    \item \textit{Non-cascade mass points:} (15, 20), (15, 30), (20, 30), (20, 40), (30, 40), (30, 50), (30, 60), (40, 50), (40, 60), (40, 70), (40, 80), (50, 60), and (50, 70) GeV
\end{itemize}
Samples were produced using the MadGraph5\_aMCatNLO event generator, for each signal mass point in the gluon-gluon fusion (ggF) and vector boson fusion (VBF) production modes of the 125 GeV Higgs boson. In the sample generation, the decays of $a$ to Standard Model particles were specified to be decays to bottom quarks or $\tau$ leptons.


\section{Cascade scenario signal studies}
% TODO: include figures
Studies of the signal phenomenology in the cascade scenario were performed to determine the viability of the $4b2\tau$ and/or $2b4\tau$ channels. 

Cross sections and branching fractions of the $4b2\tau$ and $2b4\tau$ final states were compared using cross-section predictions provided by the authors of \cite{Robens:2019kga}. For an example mass point $m_{a_2} = 80$ GeV, $m_{a_1} = 30$ GeV, the branching fractions to $4b2\tau$ is ten times larger than $2b4\tau$: $B(h \rightarrow a_1 a_2 \rightarrow 3 a_1 \rightarrow 4b2\tau) = 0.00857$, vs. $B(h \rightarrow a_1 a_2 \rightarrow 3 a_1 \rightarrow 2b4\tau) = 0.00068$. The $4b2\tau$ final state was chosen for this analysis.

An event category with three or more b-tag jets was determined to be infeasible due to low statistics in this category, due to the difficulties in reconstructing the third and fourth b jets which have very low transverse momenta $p_{T}$. Event categories with exactly 1 b-tag jet and $\geq 2$ b-tag jets will be used.

The possibility of the leading and sub-leading b-tag jets being sufficiently close in $\Delta R$ to require boosted jet reconstruction techniques was explored. In the $4b2\tau$ case, jets in the generated event were spatially matched in $\Delta R$ to the jets reconstructed with the standard AK4 algorithm which uses a cone size of $\Delta R = 0.4$ to find jet constituents. Reasonable $p_{T}$ resolution (computed as $(p_{T, \text{reconstructed}} - p_{T, \text{gen}})/ p_{T, \text{gen}}$) was seen for representative mass points.

\section{Control plots for $\mu\tau_{h}$ channel}
% TODO: include mutau control plots